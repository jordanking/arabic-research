\begin{abstract}
\label{sec:abstract}

Word embeddings are an increasingly important tool for NLP tasks that require semantic understanding of words. Methodologies and properties of English word embeddings have been extensively researched. However, little attention has been given to the production and application of Arabic word embeddings. Arabic is far more morphologically complex than English due to the many conjugations, suffixes, articles, and other grammar constructs. This has a significant effect on the training and application of Arabic word embeddings. While there are a number of techniques to break down Arabic words through lemmatization and tokenization, the quality of resulting word embeddings must be investigated to understand the effects of these transformations. In this thesis, we investigate a number of preprocessing methods and training parameterizations to establish guideline methodologies for training high quality Arabic word embeddings. Using various evaluation tasks, including a new semantic similarity task created by fluent Arabic speakers, we are able to identify training strategies that produce high quality results for each task. We then apply the best models from these experiments to a text mining task where we measure the buzz about a subject in a corpus of Arabic news documents. This application highlights the utility of word embeddings for data mining tasks. We also offer a suite of accessible open source Arabic NLP tools to increase the use of Arabic word embeddings within the research community. To summarize, the main contributions of this work include improved methodologies for training Arabic word vectors, a semantic similarity task developed by native Arabic speakers, an application of Arabic word embeddings for the buzz detection text mining task, and a python package of Arabic text processing tools.

\end{abstract}